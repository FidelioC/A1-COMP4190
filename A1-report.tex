\documentclass[12pt,fleqn]{article}
\usepackage[margin=0.5in]{geometry}
\usepackage{amsmath, amssymb}

\setlength{\mathindent}{0pt}
\allowdisplaybreaks

\title{COMP 4190 Assignment 1 Answer}
\author{Fidelio Ciandy, 7934456}
\date{}

\begin{document}
\maketitle

\section*{Problem 1}

For my first question, I have two functions with different purposes.

\begin{itemize}
  \item \textbf{TransformWord}
  \begin{itemize}
    \item This function performs the BFS.
    \item Starting from the begin word, it performs BFS by comparing the current word with the words in the word list.
    \item If the current word differs by a single letter, the new word is inserted into the queue and the sequence counter is incremented by 1.
    \item The visited word is added to a set to prevent circular loops.
    \item The loop continues until the end word is reached or the queue is empty.
  \end{itemize}

  \item \textbf{CheckAdjacentWords}
  \begin{itemize}
    \item This function will compare two strings and check whether those
    two strings differ by 1 letter or not
  \end{itemize}
\end{itemize}

\section*{Problem 2}
\section*{Problem 3}
\section*{Problem 4}
\section*{Problem 5}
\section*{Problem 6}
\begin{align*}
(a)\;& \nabla_x f(x) = \tfrac12 (A + A^T)x - b \\
    & \text{Since A is symmetric, thus, } A = A^T \\
    & \nabla_x f(x) = Ax - b \\
(b)\;& x^* : \nabla f(x^*) = 0 \\
    & Ax^* - b = 0 \\
    & Ax^* = b \\
    & A^{-1} (A x^*) = A^{-1} b \\
    & I x^* = A^{-1} b\\
    & x^* = A^{-1} b\\
(c)\;&
\end{align*}

\section*{Problem 7}
\begin{align*}
(a)\;& \nabla f(x) = 2(x - 2)\\
(b)\;& x_{k+1} = x_k - \alpha \nabla f(X_k) \\
    & x_{k+1} = x_k - \alpha 2(x-2) \\
(c)\;& x^* : \nabla f(x^*) = 0 \\
    & \text{So, } x^* = 2  \\
(d)\;& \text{if step size $\alpha$ is too large: oscillations (no convergence)} \\
    & \text{if step size $\alpha$ is too small: slow convergence} \\
(e)\;& x_{k+1} = x_k - \alpha 2(x-2) \\
    & = x_k - 2\alpha x_k + 4\alpha \\
    & = (1 - 2\alpha)x_k + 4\alpha \\
    & \text{Example, let } (1-2\alpha) = b \\
    & x_1 = x_0 b + 4 \alpha \\
    & \\
    & x_2 = x_1 b + 4 \alpha \\
    & = (x_0 b + 4 \alpha) b + 4 \alpha \\
    & = x_0 b^2 + 4 \alpha b + 4 \alpha \\
    & = x_0 b^2 + 4 \alpha (b + 1) \\
    & \\
    & x_3 = x_2 b + 4 \alpha \\
    & = (x_0 b^2 + 4 \alpha (b+1)) b + 4\alpha \\
    & = (x_0 b^2 + 4 \alpha b + 4 \alpha) b + 4 \alpha \\
    & = x_0 b^3 + 4 \alpha b^2 + 4 \alpha b + 4 \alpha \\
    & = x_0 b^3 + 4 \alpha (b^2 + b + 1) \\
    & \\
    & \text{So, in general for } x_k\\
    & x_k = x_0 b^k + 4 \alpha (b^k + b^{k-1} + .... + 1) \\
    & (b^k + b^{k-1} + .... + 1) \text{ is a geometric sum}\\
    & \text{Thus, } \\
    & x_k = x_0 b^k + 4 \alpha \left(\frac{1-b^k}{1-b}\right)\\
    & = x_0 b^k + 4 \alpha \left( \frac{1-b^k}{2 \alpha}\right)\\
    & = x_0 b^k + 2 (1-b^k)\\
    & = x_0 b^k + 2 - 2b^k\\
    & = (x_0 - 2) b^k + 2 \\
    & = (1 - 2\alpha)^k (x_0 - 2) + 2 
      \qquad \text{... sub in } b = (1-2\alpha)\\
    & \\
    & x_k \text{ converges to } x^* == x_k \text{ converges to } 2 
      \text{, since } x^* = 2\\
    & \text{Thus, } x_k - 2 \text{ converges to } 0\\ 
    & \text{In other word, } (1 - 2\alpha)^k (x_0 - 2) \rightarrow 0\\
    & \text{Furthermore, } x_0 \neq x^* \rightarrow x_0 - 2 \neq 0\\
    & \text{Therefore, } (1-2\alpha)^k \rightarrow 0\\
\end{align*}

\end{document}
